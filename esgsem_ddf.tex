
%% This is based on bare_conf.tex
%% V1.3
%% 2007/01/11
%% by Michael Shell
%% See:
%% http://www.michaelshell.org/
%% for current contact information.
%%
%% This is a skeleton file demonstrating the use of IEEEtran.cls
%% (requires IEEEtran.cls version 1.7 or later) with an IEEE conference paper.
%%
%% Support sites:
%% http://www.michaelshell.org/tex/ieeetran/
%% http://www.ctan.org/tex-archive/macros/latex/contrib/IEEEtran/
%% and
%% http://www.ieee.org/

%%*************************************************************************
%% Legal Notice:
%% This code is offered as-is without any warranty either expressed or
%% implied; without even the implied warranty of MERCHANTABILITY or
%% FITNESS FOR A PARTICULAR PURPOSE! 
%% User assumes all risk.
%% In no event shall IEEE or any contributor to this code be liable for
%% any damages or losses, including, but not limited to, incidental,
%% consequential, or any other damages, resulting from the use or misuse
%% of any information contained here.
%%
%% All comments are the opinions of their respective authors and are not
%% necessarily endorsed by the IEEE.
%%
%% This work is distributed under the LaTeX Project Public License (LPPL)
%% ( http://www.latex-project.org/ ) version 1.3, and may be freely used,
%% distributed and modified. A copy of the LPPL, version 1.3, is included
%% in the base LaTeX documentation of all distributions of LaTeX released
%% 2003/12/01 or later.
%% Retain all contribution notices and credits.
%% ** Modified files should be clearly indicated as such, including  **
%% ** renaming them and changing author support contact information. **
%%
%% File list of work: IEEEtran.cls, IEEEtran_HOWTO.pdf, bare_adv.tex,
%%                    bare_conf.tex, bare_jrnl.tex, bare_jrnl_compsoc.tex
%%*************************************************************************

\documentclass[conference,a4paper]{IEEEtran}

\usepackage{cite}
\usepackage[pdftex]{graphicx}
\usepackage[cmex10]{amsmath}
\usepackage{fixltx2e}
\usepackage{url}

\begin{document}
%
% paper title
% can use linebreaks \\ within to get better formatting as desired
\title{Bare Demo of IEEEtran.cls for Conferences}


% author names and affiliations
% use a multiple column layout for up to three different
% affiliations
\author{\IEEEauthorblockN{Your Name}
%\IEEEauthorblockA{Your Association}
}

% make the title area
\maketitle


\begin{abstract}
You can summarize the extended abstract here in few sentences. It is
also acceptable to have no abstract at all in the extended abstract
format.
\end{abstract}



\section{Introduction}
This is a skeleton file, which you can use as a starting point of your
extended abstract for T-106.5840/2011 seminar on internet of things.
For a more complete skeleton file, see bare\_conf.tex at
\url{http://www.ctan.org/tex-archive/macros/latex/contrib/IEEEtran/}. For
a thorough how-to for IEEEtran.cls, see IEEEtran\_HOWTO.pdf at that
same location.

This is an example cite \cite{liu-layland-1973} to a very classic
paper on real-time scheduling. The BibTeX database is found on
\texttt{papers.bib}. If you're not familiar with BibTeX, you can see,
\emph{e.g.},
\url{http://en.wikibooks.org/wiki/LaTeX/Bibliography_Management#Standard_templates}
on how to create BibTeX entries. Emacs has also ``Entry-Types'' menu
in the BibTeX mode.

Fig.~\ref{fig:example-1col} is an example reference to an 
one-column floating figure and Fig.~\ref{fig:example-2col} is an example
reference to a two-column figure. The width of a
single column is \texttt{21 pc} and the width of two columns including
the column spacing is \texttt{43 pc}.

You might also want to review a quick introduction to the general IEEE
two-column format guidelines, found at
\url{http://www.ieee.org/portal/cms_docs/pubs/confpubcenter/pdfs/samplems.pdf}.

The \texttt{Makefile} should suffice for compiling your \LaTeX{}
document to PDF. The available make targets are:
\begin{itemize}
\item compile --- Builds the PDF. This is also the default target.
\item info --- Outputs some information, mostly for debugging purposes.
\item clean --- Removes all intermediate files.
\item realclean --- Removes all intermediate files and the produced
  PDF.
\end{itemize}
To change the name of your document, rename this file
(\texttt{extended-abstract.tex}) and reflect that change in variable
\texttt{BASEFILE} in the \texttt{Makefile}. If you get weird,
unexplainable errors, try ``\texttt{make realclean \&\& make}''


\begin{figure}[!t]
\centering
\includegraphics[width=21pc]{Aalto_EN_SCI_21_RGB_y2.pdf}
\caption{An example 1-column figure. You can also include PNG-files
  directly. Use, \emph{e.g.}, \texttt{convert} to convert images from
  other formats. The image width is 21 pc.}
\label{fig:example-1col}
\end{figure}

\begin{figure*}[t]
\centering
\includegraphics[width=43pc]{Aalto_EN_SCI_13_RGB_b1.pdf}
\caption{An example 2-column figure. The image is in embedded PDF
  vector format which is the best graphics format for print. The image
  width is 43 pc.}
\label{fig:example-2col}
\end{figure*}

\bibliographystyle{IEEEtran}
\bibliography{papers}

\end{document}


