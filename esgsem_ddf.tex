%% This is based on bare_conf.tex
%% V1.3
%% 2007/01/11
%% by Michael Shell
%% See:
%% http://www.michaelshell.org/
%% for current contact information.
%%
%% This is a skeleton file demonstrating the use of IEEEtran.cls
%% (requires IEEEtran.cls version 1.7 or later) with an IEEE conference paper.
%%
%% Support sites:
%% http://www.michaelshell.org/tex/ieeetran/
%% http://www.ctan.org/tex-archive/macros/latex/contrib/IEEEtran/
%% and
%% http://www.ieee.org/
\documentclass[conference,a4paper]{IEEEtran}

\usepackage{cite}
\usepackage[pdftex]{graphicx}
\usepackage[cmex10]{amsmath}
\usepackage{fixltx2e}
\usepackage{url}

\begin{document}
%
% paper title
% can use linebreaks \\ within to get better formatting as desired
\title{Dynamic Dataflow}

% author names and affiliations
% use a multiple column layout for up to three different
% affiliations
\author{\IEEEauthorblockN{Risto Vuorio}
%\IEEEauthorblockA{Your Association}
}

% make the title area
\maketitle

\begin{abstract}
You can summarize the extended abstract here in few sentences. It is
also acceptable to have no abstract at all in the extended abstract
format.
\end{abstract}

\section{Introduction}
\begin{figure}[!t]
\centering
\includegraphics[width=21pc]{Aalto_EN_SCI_21_RGB_y2.pdf}
\caption{An example 1-column figure. You can also include PNG-files
  directly. Use, \emph{e.g.}, \texttt{convert} to convert images from
  other formats. The image width is 21 pc.}
\label{fig:example-1col}
\end{figure}

\begin{figure*}[t]
\centering
\includegraphics[width=43pc]{Aalto_EN_SCI_13_RGB_b1.pdf}
\caption{An example 2-column figure. The image is in embedded PDF
  vector format which is the best graphics format for print. The image
  width is 43 pc.}
\label{fig:example-2col}
\end{figure*}

\bibliographystyle{IEEEtran}
\bibliography{papers}

\end{document}


